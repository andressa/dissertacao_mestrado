\documentclass[dsc,numbers]{coppe}
\usepackage{amsmath,amssymb}
\usepackage{hyperref}
\usepackage[latin1]{inputenc}
\usepackage[brazil]{babel}
\usepackage{graphicx}% Include figure files
\usepackage{multirow}
\usepackage{indentfirst}
\usepackage{subfigure}
\usepackage{url}
\usepackage{float}
\usepackage{textcomp}
\usepackage{tikz}
\usetikzlibrary{shapes,arrows}

\makelosymbols
\makeloabbreviations

\begin{document}
  \title{
    Intelig�ncia Computacional na Avalia��o de C�digos em um Sistema Complexo de Detec��o com Desenvolvimento Colaborativo
  }
  \foreigntitle{
    Computational Intelligence in Source Code Assertion in a Complex System in a Collaborative Development Enviroment
  }
  \author{Andressa A.}{Sivolella Gomes}
  \advisor{Prof.}{Jos�}{Manoel de Seixas}{D.Sc.}

  \examiner{Prof.}{Alu�zio Fausto Ribeiro Ara�jo}{D.Sc.}
  \examiner{Prof.}{Afonso de Bediaga e Hickman}{D.Sc.}
  \examiner{Pesquisadora}{Carmen L�cia Lodi Maidantchik}{D.Sc.}
  \department{PEE}
  \date{03}{2016}

  \keyword{Minera��o de c�digos}
  \keyword{M�todos ensemble com �rvores}
  \keyword{Plataforma colaborativa}

  \maketitle

  \frontmatter
  \dedication{A todo mundo, geralz�o.}

  \chapter*{Agradecimentos}

  Gostaria de agradecer a todos.

  \begin{abstract}

  Apresenta-se, nesta tese, ...

  \end{abstract}

  \begin{foreignabstract}

  In this work, we present ...

  \end{foreignabstract}

  \tableofcontents
  \listoffigures
  \listoftables
  \printlosymbols
  \printloabbreviations

  \mainmatter
  \chapter{Introdu{\c c}\~ao}

  Segundo a norma de formata{\c c}\~ao de teses e disserta{\c c}\~oes do
  Instituto Alberto Luiz Coimbra de P\'os-gradua{\c c}\~ao e Pesquisa de
  Engenharia (COPPE), toda abreviatura deve ser definida antes de
  utilizada.\abbrev{COPPE}{Instituto Alberto Luiz Coimbra de P\'os-gradua{\c
  c}\~ao e Pesquisa de Engenharia}

  Do mesmo modo, \'e imprescind\'ivel definir os s\'imbolos, tal como o
  conjunto dos n\'umeros reais $\mathbb{R}$ e o conjunto vazio $\emptyset$.
  \symbl{$\mathbb{R}$}{Conjunto dos n\'umeros reais}
  \symbl{$\emptyset$}{Conjunto vazio}

  \chapter{Revis\~ao Bibliogr\'afica}

  Para ilustrar a completa ades\~ao ao estilo de cita{\c c}\~oes e listagem de
  refer\^encias bibliogr\'aficas, a Tabela~\ref{tab:citation} apresenta cita{\c
  c}\~oes de alguns dos trabalhos contidos na norma fornecida pela CPGP da
  COPPE, utilizando o estilo num\'erico.

  \begin{table}[h]
  \caption{Exemplos de cita{\c c}\~oes utilizando o comando padr\~ao
    \texttt{\textbackslash cite} do \LaTeX\ e
    o comando \texttt{\textbackslash citet},
    fornecido pelo pacote \texttt{natbib}.}
  \label{tab:citation}
  \centering
  {\footnotesize
  \begin{tabular}{|c|c|c|}
    \hline
    Tipo da Publica{\c c}\~ao & \verb|\cite| & \verb|\citet|\\
    \hline
    Livro & \cite{book-example} & \citet{book-example}\\
    Artigo & \cite{article-example} & \citet{article-example}\\
    Relat\'orio & \cite{techreport-example} & \citet{techreport-example}\\
    Relat\'orio & \cite{techreport-exampleIn} & \citet{techreport-exampleIn}\\
    Anais de Congresso & \cite{inproceedings-example} &
      \citet{inproceedings-example}\\
    S\'eries & \cite{incollection-example} & \citet{incollection-example}\\
    Em Livro & \cite{inbook-example} & \citet{inbook-example}\\
    Disserta{\c c}\~ao de mestrado & \cite{mastersthesis-example} &
      \citet{mastersthesis-example}\\
    Tese de doutorado & \cite{phdthesis-example} & \citet{phdthesis-example}\\
    \hline
  \end{tabular}}
  \end{table}

  \chapter{M\'etodo Proposto}
  \chapter{Resultados e Discuss\~oes}
  \chapter{Conclus\~oes}

  \backmatter
  \bibliographystyle{coppe-unsrt}
  \bibliography{dissertacao}

  \appendix
  \chapter{Algumas Demonstra{\c c}\~oes}
\end{document}
%% 
%%
%% End of file `example.tex'.
